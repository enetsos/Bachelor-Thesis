\documentclass[twoside]{supsistudent} 

% per settare noindent
\setlength{\parindent}{0pt}


% Crea un capitolo senza numerazione che pero` appare nell'indice %
\newcommand{\problemchapter}[1]{%
  \chapter*{#1}%
  \addcontentsline{toc}{chapter}{#1}%
\markboth{#1}{#1}
}

% Numerazione delle appendici secondo norma
\addto\appendix{
\renewcommand{\thesection}{\Alph{chapter}.\arabic{section}}
\renewcommand{\thesubsection}{\thesection.\arabic{subsection}}}

\setcounter{secnumdepth}{5} 	%per avere più livelli nei titoli
\setcounter{tocdepth}{5}		%per avere più livelli nell'indice


\titolo{App mobile per gestione timbrature e controllo qualità collaboratori}
\studente{Walter Sostene Losa}
\relatore{Andrea Baldassari}
\correlatore{Matteo Besenzoni}
\committente{Progect SA - Impresa Generale di Pulizie e Facility Management}
\corso{Ingegneria Informatica}
\codice{C10793}
\anno{2023/2024}



\begin{document}

\pagenumbering{alph}
\maketitle
\onehalfspacing
\frontmatter


\pagenumbering{roman}
\tableofcontents
\listoffigures					% Opzionale
\listoftables					% Opzionale

\newpage
\mainmatter
\pagenumbering{arabic}
\setcounter{page}{1}

\chapter*{Abstract}

\chapter{Progetto Assegnato}
\chapter{Introduzione}
\chapter{Motivazione e Contesto}
In questo capitolo, vengono descritte le motivazioni accademiche che stanno alla base di
questo lavoro di tesi, insieme al contesto in cui si inseriscono. In conclusione, vengono
delineati gli obiettivi che guidano lo sviluppo di questo progetto.\\
\\
Nell'ambito del corso di Bachelor in Ingegneria Informatica presso la Scuola Universitaria
Professionale della Svizzera Italiana (SUPSI), ogni studente si trova di fronte al compito di
completare un lavoro di diploma. Questo implica la scelta autonoma di un progetto tra quelli
proposti da diversi relatori. Tali progetti abbracciano una vasta gamma di settori, spaziando
tra software desktop standalone, applicazioni web, networking, intelligenza artificiale e molto
altro. Ogni studente è tenuto a sviluppare il progetto nel corso della stagione estiva, per poi
consegnarlo entro l'inizio di settembre. Lo scopo di questo progetto è applicare le tecniche,
le strategie e i metodi di sviluppo appresi durante i tre anni di studi del percorso di Bachelor.\\
\\
La scelta di svolgere questo particolare progetto è stata determinata da diversi fattori. In
particolare, l'opportunità di sviluppare un'applicazione web è stata considerata un modo
per scoprire meglio questo campo non essendo stato molto visto durante gli anni, offrendo
uno spazio ideale per l'applicazione delle competenze acquisite. L'uso di un framework
aggiornato, ha fornito un'opportunità unica per scoprire una nuova tecnologia ancora non molto
diffusa nei sfotware odierni.\\
\\
Il progetto è seguito dai relatori Baldassari Andrea e Matteo Besenzoni, entrambi docenti e ricer-
catori presso la SUPSI. In particolare, il docente Andrea Baldassari ha fornito i requisiti e le specifiche man mano che il lavoro procedeva.

\section{Obiettivo}
L'obiettivo di questo progetto è sviluppare un'applicazione web per la gestione delle timbrature
e il controllo qualità dei collaboratori di Progect SA. L'applicazione deve permettere ai
collaboratori di timbrare in entrata e in uscita, di inserire le attività svolte durante la giornata
e di caricare le foto dei lavori eseguiti. Inoltre, l'applicazione deve permettere ai responsabili
di controllo qualità di visualizzare le timbrature e le attività svolte dai collaboratori, di
valutare la qualità del lavoro svolto e di fornire feedback ai collaboratori.\\
\\

\chapter{Problema}
\chapter{Stato dell'arte}
\chapter{Approccio al problema}
\chapter{Progettazione}
\chapter{Architettura}
\chapter{Implementazione}
\chapter{Test}
\chapter{Risultati}
\chapter{Conclusioni}

\bibliographystyle{unsrt}
\bibliography{bibliografia}
\end{document}
