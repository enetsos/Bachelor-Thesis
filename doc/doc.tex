\documentclass[twoside]{supsistudent} 

% per settare noindent
\setlength{\parindent}{0pt}


% Crea un capitolo senza numerazione che pero` appare nell'indice %
\newcommand{\problemchapter}[1]{%
  \chapter*{#1}%
  \addcontentsline{toc}{chapter}{#1}%
\markboth{#1}{#1}
}

% Numerazione delle appendici secondo norma
\addto\appendix{
\renewcommand{\thesection}{\Alph{chapter}.\arabic{section}}
\renewcommand{\thesubsection}{\thesection.\arabic{subsection}}}

\setcounter{secnumdepth}{5} 	%per avere più livelli nei titoli
\setcounter{tocdepth}{5}		%per avere più livelli nell'indice


\titolo{App mobile per gestione timbrature e controllo qualità collaboratori}
\studente{Walter Sostene Losa}
\relatore{Andrea Baldassari}
\correlatore{Matteo Besenzoni}
\committente{Progect SA - Impresa Generale di Pulizie e Facility Management}
\corso{Ingegneria Informatica}
\codice{C10793}
\anno{2023/2024}



\begin{document}

\pagenumbering{alph}
\maketitle
\onehalfspacing
\frontmatter


\pagenumbering{roman}
\tableofcontents
\listoffigures					% Opzionale
\listoftables					% Opzionale

\newpage
\mainmatter
\pagenumbering{arabic}
\setcounter{page}{1}

\chapter*{Abstract}

\chapter{Progetto Assegnato}
\chapter{Introduzione}
\chapter{Motivazione e Contesto}
\chapter{Problema}
\chapter{Stato dell'arte}
\chapter{Approccio al problema}
\chapter{Progettazione}
\chapter{Architettura}
\chapter{Implementazione}
\chapter{Test}
\chapter{Risultati}
\chapter{Conclusioni}

\bibliographystyle{unsrt}
\bibliography{bibliografia}
\end{document}
