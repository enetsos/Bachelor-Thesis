\documentclass[twoside]{supsistudent} 

% per settare noindent
\setlength{\parindent}{0pt}


% Crea un capitolo senza numerazione che pero` appare nell'indice %
\newcommand{\problemchapter}[1]{%
  \chapter*{#1}%
  \addcontentsline{toc}{chapter}{#1}%
\markboth{#1}{#1}
}

% Numerazione delle appendici secondo norma
\addto\appendix{
\renewcommand{\thesection}{\Alph{chapter}.\arabic{section}}
\renewcommand{\thesubsection}{\thesection.\arabic{subsection}}}

\setcounter{secnumdepth}{5} 	%per avere più livelli nei titoli
\setcounter{tocdepth}{5}		%per avere più livelli nell'indice


\titolo{App mobile per gestione timbrature e controllo qualità collaboratori}
\studente{Walter Sostene Losa}
\relatore{Andrea Baldassari}
\correlatore{Matteo Besenzoni}
\committente{Progect SA - Impresa Generale di Pulizie e Facility Management}
\corso{Ingegneria Informatica}
\codice{C10793}
\anno{2023/2024}



\begin{document}

\pagenumbering{alph}
\maketitle
\onehalfspacing
\frontmatter


\pagenumbering{roman}
\tableofcontents
\listoffigures					% Opzionale
\listoftables					% Opzionale

\newpage
\mainmatter
\pagenumbering{arabic}
\setcounter{page}{1}

\chapter*{Abstract}

\chapter{Progetto Assegnato}

Il partner di progetto, un'importante azienda attiva nei servizi di pulizia e facility management, necessita di un sistema che permetta la gestione e il controllo qualità dei servizi erogati presso i clienti. Il sistema, accessibile tramite applicazione mobile, sarà utilizzato principalmente da tre attori: Dipendenti, Responsabili e Clienti.

\section{Funzionalità per i Dipendenti}

I Dipendenti avranno a disposizione le seguenti funzionalità:
\begin{itemize}
  \item \textbf{Registrazione delle presenze}: I dipendenti possono registrare (timbrare) il loro arrivo e la loro partenza presso i clienti da cui si recano, tramite QR code o tag NFC dedicato.
  \item \textbf{Segnalazioni di anomalie}: Possono fare segnalazioni, allegando anche foto e video, presso i clienti, in caso di anomalie riscontrate durante il servizio.
  \item \textbf{Ordini di attrezzature e prodotti}: Possono fare un ordine delle attrezzature e dei prodotti che sono finiti o che sono in esaurimento, facilitando la gestione del materiale necessario per svolgere le loro mansioni.
\end{itemize}

\section{Funzionalità per i Responsabili}

I Responsabili avranno a disposizione le seguenti funzionalità:
\begin{itemize}
  \item \textbf{Verifica delle timbrature}: Possono verificare le timbrature dei dipendenti di loro responsabilità ai fini di controllo qualità, assicurando che i servizi vengano svolti nei tempi previsti.
  \item \textbf{Segnalazioni di controllo qualità}: Possono fare segnalazioni per controllo qualità, direttamente al dipendente ed indicando il cliente, permettendo una rapida risoluzione di eventuali problemi riscontrati.
\end{itemize}

\section{Funzionalità per i Clienti}

I Clienti avranno a disposizione le seguenti funzionalità:
\begin{itemize}
  \item \textbf{Segnalazioni di criticità}: Possono segnalare criticità nel servizio ricevuto, offrendo un feedback diretto all'azienda per migliorare la qualità del servizio.
  \item \textbf{Sondaggi sulla qualità del servizio}: Possono essere sottoposti a sondaggi sulla qualità del servizio ricevuto, fornendo dati utili per il miglioramento continuo.
\end{itemize}

\section{Obiettivi del Sistema}

Il sistema è progettato per raggiungere i seguenti obiettivi:
\begin{itemize}
  \item \textbf{Miglioramento della gestione delle presenze}: Tramite l'uso di tecnologie come QR code e tag NFC, il sistema assicura una registrazione precisa e affidabile delle presenze dei dipendenti.
  \item \textbf{Aumento della trasparenza e della comunicazione}: Le funzionalità di segnalazione per dipendenti, responsabili e clienti aumentano la trasparenza e migliorano la comunicazione tra le parti coinvolte.
  \item \textbf{Ottimizzazione delle risorse}: La possibilità di ordinare attrezzature e prodotti direttamente tramite l'app facilita la gestione delle risorse, assicurando che i dipendenti abbiano sempre a disposizione il materiale necessario.
  \item \textbf{Controllo qualità continuo}: Le funzionalità di verifica e segnalazione per i responsabili permettono un controllo qualità continuo, garantendo un elevato standard del servizio erogato.
\end{itemize}


\chapter{Introduzione}

Il progetto nasce dalla necessità di Progect SA, un'importante azienda attiva nei servizi di pulizia e facility management, di migliorare la gestione e il controllo qualità dei servizi erogati presso i clienti. L'azienda desidera un sistema che permetta ai dipendenti di registrare le presenze presso i clienti, di segnalare anomalie riscontrate durante il servizio e di fare ordini di attrezzature e prodotti. I responsabili devono poter verificare le timbrature dei dipendenti di loro responsabilità e fare segnalazioni per controllo qualità. I clienti devono poter segnalare criticità nel servizio ricevuto e essere sottoposti a sondaggi sulla qualità del servizio. Il sistema deve garantire una registrazione precisa e affidabile delle presenze, aumentare la trasparenza e migliorare la comunicazione tra le parti coinvolte, facilitare la gestione delle risorse e assicurare un controllo qualità continuo.\\
\\
Il sistema sarà accessibile tramite applicazione mobile e sarà utilizzato principalmente da tre attori: Dipendenti, Responsabili e Clienti. I Dipendenti potranno registrare le presenze, fare segnalazioni di anomalie e fare ordini di attrezzature e prodotti. I Responsabili potranno verificare le timbrature dei dipendenti e fare segnalazioni per controllo qualità. I Clienti potranno segnalare criticità e essere sottoposti a sondaggi sulla qualità del servizio.\\
\\
Il Capitolo 2 del documento presenta un’analisi delle motivazioni e del contesto che hanno guidato lo sviluppo del progetto. Inoltre, vengono introdotti in modo preliminare gli obiettivi che costituiscono il nucleo centrale di questo lavoro.\\
\\
todo: aggiungere una breve spiegazione di ogni capitolo

\chapter{Motivazione e Contesto}

In questo capitolo, vengono descritte le motivazioni accademiche che stanno alla base di
questo lavoro di tesi, insieme al contesto in cui si inseriscono. In conclusione, vengono
delineati gli obiettivi che guidano lo sviluppo di questo progetto.\\
\\
Nell'ambito del corso di Bachelor in Ingegneria Informatica presso la Scuola Universitaria
Professionale della Svizzera Italiana (SUPSI), ogni studente si trova di fronte al compito di
completare un lavoro di diploma. Questo implica la scelta autonoma di un progetto tra quelli
proposti da diversi relatori. Tali progetti abbracciano una vasta gamma di settori, spaziando
tra software desktop standalone, applicazioni web, networking, intelligenza artificiale e molto
altro. Ogni studente è tenuto a sviluppare il progetto nel corso della stagione estiva, per poi
consegnarlo entro l'inizio di settembre. Lo scopo di questo progetto è applicare le tecniche,
le strategie e i metodi di sviluppo appresi durante i tre anni di studi del percorso di Bachelor.\\
\\
La scelta di svolgere questo particolare progetto è stata determinata da diversi fattori. In
particolare, l'opportunità di sviluppare un'applicazione web è stata considerata un modo
per scoprire meglio questo campo non essendo stato molto visto durante gli anni, offrendo
uno spazio ideale per l'applicazione delle competenze acquisite. L'uso di un framework
aggiornato, ha fornito un'opportunità unica per scoprire una nuova tecnologia ancora non molto
diffusa nei sfotware odierni.\\
\\
Il progetto è seguito dai relatori Baldassari Andrea e Matteo Besenzoni, entrambi docenti e ricer-
catori presso la SUPSI. In particolare, il docente Andrea Baldassari ha fornito i requisiti e le specifiche man mano che il lavoro procedeva.

\section{Obiettivo}
L'obiettivo di questo progetto è sviluppare un'applicazione web per la gestione delle timbrature
e il controllo qualità dei collaboratori di Progect SA. L'applicazione deve permettere ai
collaboratori di timbrare in entrata e in uscita, di inserire le attività svolte durante la giornata
e di caricare le foto dei lavori eseguiti. Inoltre, l'applicazione deve permettere ai responsabili
di controllo qualità di visualizzare le timbrature e le attività svolte dai collaboratori, di
valutare la qualità del lavoro svolto e di fornire feedback ai collaboratori.\\
\\

\chapter{Problema}
Il problema affrontato nel contesto del progetto riguarda la necessità di sviluppare un sistema completo per la gestione delle timbrature e il controllo della qualità dei servizi offerti da un'azienda attiva nel settore della pulizia e del facility management. Attualmente, l'azienda si trova a dover affrontare diverse sfide legate alla gestione e al monitoraggio dei propri collaboratori e dei servizi erogati, tra cui:

\begin{itemize}
  \item \textbf{Mancanza di un sistema di tracciamento efficiente}: L'assenza di un sistema integrato per la registrazione delle timbrature dei dipendenti presso i vari clienti rende difficile il monitoraggio in tempo reale delle attività lavorative e il controllo della qualità dei servizi erogati.

  \item \textbf{Difficoltà nella gestione delle segnalazioni}: Attualmente, le segnalazioni relative a anomalie o criticità del servizio devono essere gestite manualmente, il che comporta inefficienze operative e ritardi nella risoluzione dei problemi.

  \item \textbf{Carenza di trasparenza nelle comunicazioni}: I canali di comunicazione tra dipendenti, responsabili e clienti non sono ottimizzati, portando a una scarsa trasparenza e a possibili incomprensioni riguardo alle prestazioni del servizio e alle necessità dei clienti.

  \item \textbf{Inadeguatezza nella gestione delle scorte di attrezzature e prodotti}: La mancanza di un sistema automatizzato per la gestione delle richieste di rifornimento di attrezzature e prodotti può causare ritardi nel rifornimento, con conseguente impatto negativo sulla qualità del servizio.

  \item \textbf{Limitazioni nell'integrazione tecnologica}: L'integrazione con dispositivi mobili e tecnologie come NFC e QR code non è stata pienamente sfruttata, limitando la possibilità di utilizzare strumenti digitali avanzati per il controllo della qualità e la gestione delle attività lavorative.

\end{itemize}


\chapter{Stato dell'arte}


\chapter{Approccio al problema}
In questo capitolo, esamineremo in dettaglio l'approccio adottato per sviluppare una Progressive Web App (PWA) dedicata alla gestione e al controllo qualità dei servizi di pulizia e facility management. Illustreremo le scelte effettuate per quanto riguarda la gestione del progetto, i linguaggi di programmazione, i framework selezionati e l'integrazione con i componenti dei dispositivi mobili. Ogni decisione tecnica è stata presa con l'obiettivo di garantire prestazioni elevate, facilità di manutenzione e un'esperienza utente ottimale.

\section{Gestione del progetto}

La gestione del progetto è stata condotta utilizzando una metodologia Agile, in particolare il framework Scrum. Questa scelta è stata dettata dalla necessità di mantenere un'elevata flessibilità durante lo sviluppo, permettendo al team di adattarsi rapidamente ai cambiamenti nei requisiti del cliente. Ogni settimana è stato condotto uno sprint, al termine del quale si teneva una riunione con i committenti per presentare i progressi attraverso demo, raccogliere feedback e pianificare le attività per la settimana successiva.

Per la gestione del codice sorgente e delle attività del progetto, è stata utilizzata la piattaforma GitLab. GitLab ha offerto strumenti integrati per il versionamento del codice, la gestione delle issue, la pianificazione degli sprint e l'integrazione continua. Questi strumenti hanno facilitato una stretta collaborazione all'interno del team e una gestione efficace del progetto, assicurando che il codice fosse sempre aggiornato e che ogni membro del team fosse informato sugli sviluppi.

\section{Linguaggi di Programmazione}

\subsection{Backend}

Per lo sviluppo del backend, abbiamo scelto di utilizzare Node.js in combinazione con il framework Express e l'ORM Sequelize. La scelta di Node.js è stata guidata da diversi fattori chiave:

\begin{itemize}
  \item \textbf{Asincronicità e Performance}: Node.js è basato su un modello di I/O non bloccante, che consente di gestire un elevato numero di richieste simultanee con un'efficienza notevole. Questo è particolarmente vantaggioso per un'applicazione che deve interagire con dispositivi mobili e gestire numerose operazioni in tempo reale, come l'acquisizione di dati dai sensori dei dispositivi.
  \item \textbf{Ampio Ecosistema}: Node.js dispone di un vasto ecosistema di librerie e moduli disponibili attraverso npm (Node Package Manager). Questo ci ha permesso di accelerare lo sviluppo utilizzando pacchetti preesistenti per funzionalità comuni, riducendo il tempo necessario per implementare funzioni complesse.
  \item \textbf{Scalabilità}: La natura asincrona di Node.js lo rende particolarmente adatto per costruire applicazioni scalabili. Per un sistema come quello proposto, che potrebbe essere utilizzato da numerosi utenti contemporaneamente, la capacità di scalare orizzontalmente è cruciale.
\end{itemize}

Express, un framework minimalista per Node.js, è stato scelto per la sua semplicità e flessibilità nella gestione delle API RESTful. Express consente di strutturare l'applicazione in modo modulare, facilitando la manutenzione e l'espansione del codice nel tempo. Inoltre, con l'uso di Sequelize, un ORM per Node.js, è stato possibile gestire le interazioni con il database MySQL in modo sicuro ed efficiente, mappando le tabelle del database su oggetti JavaScript e semplificando le operazioni CRUD (Create, Read, Update, Delete).

\subsection{Frontend}

Per il frontend, è stato scelto React come framework principale per lo sviluppo dell'interfaccia utente. Le motivazioni alla base di questa scelta sono molteplici:

\begin{itemize}
  \item \textbf{Componentizzazione}: React adotta un approccio basato su componenti, che permette di costruire interfacce utente modulari e riutilizzabili. Questa modularità non solo facilita lo sviluppo, ma rende anche più semplice la manutenzione e l'aggiornamento dell'applicazione nel tempo.
  \item \textbf{Performance e Virtual DOM}: React utilizza un Virtual DOM per ottimizzare il rendering dell'interfaccia utente. Questo riduce al minimo le operazioni di manipolazione del DOM reale, migliorando le prestazioni, soprattutto su dispositivi mobili, dove la potenza di calcolo può essere limitata.
  \item \textbf{Ecosistema e Community}: La vasta community di React e l'ampio ecosistema di librerie di supporto hanno fornito strumenti utili e best practices che hanno accelerato lo sviluppo e migliorato la qualità del codice.
\end{itemize}

Inoltre, per migliorare ulteriormente la velocità di sviluppo e la coerenza dell'interfaccia utente, è stata utilizzata la libreria \textbf{Ant Design (Antd)}. Antd è un set di componenti UI predefiniti per React, che offre una vasta gamma di elementi grafici stilisticamente coerenti e pronti all'uso. L'adozione di Antd ha permesso al team di:

\begin{itemize}
  \item \textbf{Accelerare lo Sviluppo}: Utilizzando componenti predefiniti di alta qualità, il team ha potuto concentrarsi maggiormente sulle logiche di business piuttosto che sulla progettazione e implementazione di elementi grafici complessi.
  \item \textbf{Mantenere una UI Coerente}: Antd garantisce che l'interfaccia utente sia esteticamente coerente e intuitiva, migliorando l'esperienza utente complessiva.
  \item \textbf{Facile Personalizzazione}: Sebbene Antd offra componenti predefiniti, è altamente personalizzabile, consentendo al team di adattare l'aspetto e il comportamento dei componenti alle specifiche esigenze del progetto.
\end{itemize}

L'uso di React, insieme a Antd, ha inoltre facilitato l'integrazione con le API dei dispositivi, come la fotocamera e i sensori di posizione, rendendo possibile la realizzazione di funzionalità critiche come la scansione di QR code e la tracciabilità GPS.

\section{Integrazione con Componenti del Dispositivo}

Uno degli aspetti più innovativi dell'applicazione è la sua capacità di interagire direttamente con i componenti hardware dei dispositivi mobili. Grazie a librerie specifiche di React, l'applicazione può accedere alla fotocamera per scattare foto e scansionare codici QR, nonché raccogliere dati GPS per tracciare la posizione dei lavoratori in tempo reale. Questo livello di integrazione è stato fondamentale per garantire che il sistema potesse offrire funzionalità di controllo qualità avanzate, come la verifica dell'ubicazione del personale sul campo e la documentazione delle attività svolte.

\section{Toolkits}

Oltre ai linguaggi e framework descritti, sono stati utilizzati diversi toolkits per supportare lo sviluppo:

\begin{itemize}
  \item \textbf{npm}: Il gestore di pacchetti npm è stato fondamentale per la gestione delle dipendenze e per l'installazione di librerie esterne che hanno velocizzato lo sviluppo.
  \item \textbf{Webpack}: Webpack è stato utilizzato per il bundling dei file JavaScript, ottimizzando le performance del frontend e migliorando i tempi di caricamento dell'applicazione.
  \item \textbf{GitLab CI/CD}: La pipeline CI/CD di GitLab ha automatizzato il processo di testing e deployment, garantendo che ogni modifica al codice fosse adeguatamente verificata prima di essere distribuita nell'ambiente di produzione.
\end{itemize}

L'insieme di questi strumenti ha permesso di creare un ambiente di sviluppo efficiente e affidabile, facilitando il rilascio continuo di nuove funzionalità e miglioramenti.

\chapter{Progettazione}

In questo capitolo viene descritta la fase di progettazione dell’applicativo, partendo dall'analisi dei requisiti necessari per poi passare alla descrizione delle funzionalità principali e dei mockup delle interfacce che saranno utilizzate dagli utenti.

\section{Requisiti}

Il sistema, accessibile tramite app, avrà principalmente quattro tipi di utenti: amministratori, dipendenti, responsabili (supervisori), e clienti. Di seguito sono elencati i requisiti specifici per ciascuno di questi ruoli.

\subsection{Requisiti per gli Amministratori}

Gli amministratori hanno il compito di gestire l'intero sistema, comprese le operazioni di gestione utenti e forniture. I requisiti specifici per questo ruolo includono:

\begin{itemize}
  \item Creare, modificare e gestire gli account degli utenti (dipendenti, responsabili e clienti).
  \item Aggiungere, modificare e gestire le forniture disponibili nel database.
  \item Monitorare tutte le segnalazioni di controllo qualità effettuate dai clienti.
  \item Visualizzare e gestire tutti i servizi svolti dai dipendenti.
\end{itemize}

\subsection{Requisiti per i Dipendenti}

I dipendenti sono coloro che eseguono le attività operative presso i clienti. I requisiti specifici per questo ruolo includono:

\begin{itemize}
  \item Registrare (timbrare) il loro arrivo e la loro partenza presso i clienti utilizzando QR code.
  \item Effettuare segnalazioni in caso di anomalie riscontrate presso i clienti.
  \item Ordinare attrezzature e prodotti che sono finiti o in esaurimento.
\end{itemize}

\subsection{Requisiti per i Responsabili (Supervisori)}

I responsabili, che sono dipendenti con funzioni aggiuntive, hanno il compito di controllare la qualità del lavoro svolto dai dipendenti sotto la loro responsabilità. I requisiti specifici per questo ruolo includono:

\begin{itemize}
  \item Verificare le timbrature dei dipendenti di loro responsabilità ai fini del controllo qualità.
\end{itemize}

\subsection{Requisiti per i Clienti}

I clienti sono coloro che ricevono i servizi e possono fornire feedback sulla loro qualità. I requisiti specifici per questo ruolo includono:

\begin{itemize}
  \item Segnalare criticità nel servizio ricevuto.
\end{itemize}


\section{Mockup delle Interfacce}

In questa sezione vengono descritti i mockup delle interfacce dell’applicazione, con particolare attenzione alle schermate più rilevanti per ciascun attore coinvolto (dipendenti, responsabili e clienti).

\subsection{Pagina di accesso}
\subsection{Pagina di registrazione delle timbrature}
\subsection{Pagina di gestione delle segnalazioni}
\subsection{Pagina di controllo qualità per i responsabili}
\subsection{Pagina di feedback dei clienti}



\chapter{Architettura}
\chapter{Implementazione}
\chapter{Test}
\chapter{Risultati}
\chapter{Conclusioni}


\cite{1705631}
\bibliographystyle{unsrt}
\bibliography{bibliografia}
\end{document}
