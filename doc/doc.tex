\documentclass[twoside]{supsistudent} 

% per settare noindent
\setlength{\parindent}{0pt}


% Crea un capitolo senza numerazione che pero` appare nell'indice %
\newcommand{\problemchapter}[1]{%
  \chapter*{#1}%
  \addcontentsline{toc}{chapter}{#1}%
\markboth{#1}{#1}
}

% Numerazione delle appendici secondo norma
\addto\appendix{
\renewcommand{\thesection}{\Alph{chapter}.\arabic{section}}
\renewcommand{\thesubsection}{\thesection.\arabic{subsection}}}

\setcounter{secnumdepth}{5} 	%per avere più livelli nei titoli
\setcounter{tocdepth}{5}		%per avere più livelli nell'indice


\titolo{App mobile per gestione timbrature e controllo qualità collaboratori}
\studente{Walter Sostene Losa}
\relatore{Andrea Baldassari}
\correlatore{Matteo Besenzoni}
\committente{Progect SA - Impresa Generale di Pulizie e Facility Management}
\corso{Ingegneria Informatica}
\codice{C10793}
\anno{2023/2024}



\begin{document}

\pagenumbering{alph}
\maketitle
\onehalfspacing
\frontmatter


\pagenumbering{roman}
\tableofcontents
\listoffigures					% Opzionale
\listoftables					% Opzionale

\newpage
\mainmatter
\pagenumbering{arabic}
\setcounter{page}{1}

\chapter*{Abstract}

\chapter{Progetto Assegnato}

Il partner di progetto, un'importante azienda attiva nei servizi di pulizia e facility management, necessita di un sistema che permetta la gestione e il controllo qualità dei servizi erogati presso i clienti. Il sistema, accessibile tramite applicazione mobile, sarà utilizzato principalmente da tre attori: Dipendenti, Responsabili e Clienti.

\section{Funzionalità per i Dipendenti}

I Dipendenti avranno a disposizione le seguenti funzionalità:
\begin{itemize}
  \item \textbf{Registrazione delle presenze}: I dipendenti possono registrare (timbrare) il loro arrivo e la loro partenza presso i clienti da cui si recano, tramite QR code o tag NFC dedicato.
  \item \textbf{Segnalazioni di anomalie}: Possono fare segnalazioni, allegando anche foto e video, presso i clienti, in caso di anomalie riscontrate durante il servizio.
  \item \textbf{Ordini di attrezzature e prodotti}: Possono fare un ordine delle attrezzature e dei prodotti che sono finiti o che sono in esaurimento, facilitando la gestione del materiale necessario per svolgere le loro mansioni.
\end{itemize}

\section{Funzionalità per i Responsabili}

I Responsabili avranno a disposizione le seguenti funzionalità:
\begin{itemize}
  \item \textbf{Verifica delle timbrature}: Possono verificare le timbrature dei dipendenti di loro responsabilità ai fini di controllo qualità, assicurando che i servizi vengano svolti nei tempi previsti.
  \item \textbf{Segnalazioni di controllo qualità}: Possono fare segnalazioni per controllo qualità, direttamente al dipendente ed indicando il cliente, permettendo una rapida risoluzione di eventuali problemi riscontrati.
\end{itemize}

\section{Funzionalità per i Clienti}

I Clienti avranno a disposizione le seguenti funzionalità:
\begin{itemize}
  \item \textbf{Segnalazioni di criticità}: Possono segnalare criticità nel servizio ricevuto, offrendo un feedback diretto all'azienda per migliorare la qualità del servizio.
  \item \textbf{Sondaggi sulla qualità del servizio}: Possono essere sottoposti a sondaggi sulla qualità del servizio ricevuto, fornendo dati utili per il miglioramento continuo.
\end{itemize}

\section{Obiettivi del Sistema}

Il sistema è progettato per raggiungere i seguenti obiettivi:
\begin{itemize}
  \item \textbf{Miglioramento della gestione delle presenze}: Tramite l'uso di tecnologie come QR code e tag NFC, il sistema assicura una registrazione precisa e affidabile delle presenze dei dipendenti.
  \item \textbf{Aumento della trasparenza e della comunicazione}: Le funzionalità di segnalazione per dipendenti, responsabili e clienti aumentano la trasparenza e migliorano la comunicazione tra le parti coinvolte.
  \item \textbf{Ottimizzazione delle risorse}: La possibilità di ordinare attrezzature e prodotti direttamente tramite l'app facilita la gestione delle risorse, assicurando che i dipendenti abbiano sempre a disposizione il materiale necessario.
  \item \textbf{Controllo qualità continuo}: Le funzionalità di verifica e segnalazione per i responsabili permettono un controllo qualità continuo, garantendo un elevato standard del servizio erogato.
\end{itemize}


\chapter{Introduzione}

Il progetto nasce dalla necessità di Progect SA, un'importante azienda attiva nei servizi di pulizia e facility management, di migliorare la gestione e il controllo qualità dei servizi erogati presso i clienti. L'azienda desidera un sistema che permetta ai dipendenti di registrare le presenze presso i clienti, di segnalare anomalie riscontrate durante il servizio e di fare ordini di attrezzature e prodotti. I responsabili devono poter verificare le timbrature dei dipendenti di loro responsabilità e fare segnalazioni per controllo qualità. I clienti devono poter segnalare criticità nel servizio ricevuto e essere sottoposti a sondaggi sulla qualità del servizio. Il sistema deve garantire una registrazione precisa e affidabile delle presenze, aumentare la trasparenza e migliorare la comunicazione tra le parti coinvolte, facilitare la gestione delle risorse e assicurare un controllo qualità continuo.\\
\\
Il sistema sarà accessibile tramite applicazione mobile e sarà utilizzato principalmente da tre attori: Dipendenti, Responsabili e Clienti. I Dipendenti potranno registrare le presenze, fare segnalazioni di anomalie e fare ordini di attrezzature e prodotti. I Responsabili potranno verificare le timbrature dei dipendenti e fare segnalazioni per controllo qualità. I Clienti potranno segnalare criticità e essere sottoposti a sondaggi sulla qualità del servizio.\\
\\
Il Capitolo 2 del documento presenta un’analisi delle motivazioni e del contesto che hanno guidato lo sviluppo del progetto. Inoltre, vengono introdotti in modo preliminare gli obiettivi che costituiscono il nucleo centrale di questo lavoro.\\
\\
todo: aggiungere una breve spiegazione di ogni capitolo

\chapter{Motivazione e Contesto}

In questo capitolo, vengono descritte le motivazioni accademiche che stanno alla base di
questo lavoro di tesi, insieme al contesto in cui si inseriscono. In conclusione, vengono
delineati gli obiettivi che guidano lo sviluppo di questo progetto.\\
\\
Nell'ambito del corso di Bachelor in Ingegneria Informatica presso la Scuola Universitaria
Professionale della Svizzera Italiana (SUPSI), ogni studente si trova di fronte al compito di
completare un lavoro di diploma. Questo implica la scelta autonoma di un progetto tra quelli
proposti da diversi relatori. Tali progetti abbracciano una vasta gamma di settori, spaziando
tra software desktop standalone, applicazioni web, networking, intelligenza artificiale e molto
altro. Ogni studente è tenuto a sviluppare il progetto nel corso della stagione estiva, per poi
consegnarlo entro l'inizio di settembre. Lo scopo di questo progetto è applicare le tecniche,
le strategie e i metodi di sviluppo appresi durante i tre anni di studi del percorso di Bachelor.\\
\\
La scelta di svolgere questo particolare progetto è stata determinata da diversi fattori. In
particolare, l'opportunità di sviluppare un'applicazione web è stata considerata un modo
per scoprire meglio questo campo non essendo stato molto visto durante gli anni, offrendo
uno spazio ideale per l'applicazione delle competenze acquisite. L'uso di un framework
aggiornato, ha fornito un'opportunità unica per scoprire una nuova tecnologia ancora non molto
diffusa nei sfotware odierni.\\
\\
Il progetto è seguito dai relatori Baldassari Andrea e Matteo Besenzoni, entrambi docenti e ricer-
catori presso la SUPSI. In particolare, il docente Andrea Baldassari ha fornito i requisiti e le specifiche man mano che il lavoro procedeva.

\section{Obiettivo}
L'obiettivo di questo progetto è sviluppare un'applicazione web per la gestione delle timbrature
e il controllo qualità dei collaboratori di Progect SA. L'applicazione deve permettere ai
collaboratori di timbrare in entrata e in uscita, di inserire le attività svolte durante la giornata
e di caricare le foto dei lavori eseguiti. Inoltre, l'applicazione deve permettere ai responsabili
di controllo qualità di visualizzare le timbrature e le attività svolte dai collaboratori, di
valutare la qualità del lavoro svolto e di fornire feedback ai collaboratori.\\
\\

\chapter{Problema}
\chapter{Stato dell'arte}
\chapter{Approccio al problema}
\chapter{Progettazione}
\chapter{Architettura}
\chapter{Implementazione}
\chapter{Test}
\chapter{Risultati}
\chapter{Conclusioni}

\bibliographystyle{unsrt}
\bibliography{bibliografia}
\end{document}
